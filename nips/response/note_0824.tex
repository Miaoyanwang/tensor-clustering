\documentclass[11pt]{article}
\usepackage{lscape}
\usepackage{amsmath,amssymb}
\usepackage{amsthm}
\usepackage{float}
\usepackage{booktabs}
\usepackage{graphicx}
\usepackage{comment}
\usepackage{bm}
\usepackage{gensymb}
\allowdisplaybreaks[4]
\usepackage{geometry}
\geometry{margin=1in}
\usepackage{setspace}
\usepackage{siunitx}
\usepackage{enumitem}
\usepackage{dsfont}

\usepackage{graphics}
\usepackage[noend]{algorithmic}
\usepackage[utf8x]{inputenc}
\usepackage{bm}

\usepackage{hyperref}
\hypersetup{
    colorlinks=true,
    citecolor = blue,
    linkcolor=blue,
    filecolor=magenta,           
    urlcolor=cyan,
}


\theoremstyle{plain}
\newtheorem{thm}{Theorem}
\newtheorem{lem}{Lemma}
\newtheorem{prop}{Proposition}
\newtheorem{pro}{Property}
\newtheorem{cor}{Corollary}
\newtheorem{ass}{Assumption}

\theoremstyle{definition}
\newtheorem{defn}{Definition}
\newtheorem{exmp}{Example}
\newtheorem{rmk}{Remark}



\usepackage[labelfont=bf]{caption}

\setcounter{table}{1}
 % \usepackage[labelformat=empty]{ caption}
\usepackage{multirow}
\usepackage{tabularx}

\def\fixme#1#2{\textbf{[FIXME (#1): #2]}}

 

\newcommand*{\KeepStyleUnderBrace}[1]{%f
  \mathop{%
    \mathchoice
    {\underbrace{\displaystyle#1}}%
    {\underbrace{\textstyle#1}}%
    {\underbrace{\scriptstyle#1}}%
    {\underbrace{\scriptscriptstyle#1}}%
  }\limits
}
% \usepackage[initials]{amsrefs}
%\usepackage{amsaddr}
\usepackage{mathtools}
\mathtoolsset{showonlyrefs=true}


\usepackage{hyperref}
\hypersetup{colorlinks=true}
\usepackage[parfill]{parskip}
\usepackage{bm}
\onehalfspacing
%%%%%%%%%%%%%%%%%%%%%%%%%%%%%%%%%%%%%%%%%%%%%%%%%%%%%%%%%%%%%%%%%%%%%
%%             Math Symbols
%%%%%%%%%%%%%%%%%%%%%%%%%%%%%%%%%%%%%%%%%%%%%%%%%%%%%%%%%%%%%%%%%%%%%

%%               Bold Math
\input macros.tex
\def\refer#1{\emph{\color{blue}#1}}

\usepackage{sectsty}
\sectionfont{\fontsize{12}{15}\selectfont}

\newcommand*{\QEDB}{\hfill\ensuremath{\square}}%


%\author{%
%Yuchen Zeng \\
%University of Wisconsin -- Madison\\
 %\texttt{yzeng58@wisc.edu} \\
%\And
%Miaoyan Wang \\
%University of Wisconsin -- Madison\\
%\texttt{miaoyan.wang@wisc.edu} \\
%}

\begin{document}
\begin{center}
{\bf \large Proof sketch}
\end{center}

Without loss of generality, assume $\mP_k=\mI_k$ for all $k=1,\ldots,K$. 
 

First show that: Suppose $\text{MCR}(\mM_k,\ \hat \mM_k)\geq \varepsilon$, then there exist $r_k\neq r'_k\in [R_k]$ such that at least one of the following two events holds: (1) $\mD_{r_kr_k} \geq \mD_{r_kr'_k} \geq {\varepsilon \over R_k^2}$, or (2) $\mD_{r_kr_k} \geq \mD_{r'_kr_k} \geq {\varepsilon \over R_k^2}$. We provide the proof when the case (2) holds. The proof under case (1) can be obtained similarly. 

Suppose (2) does not hold. Then for any $r_k\in[K]$, $\mD_{r_kr_k}$

Second show that: Suppose $\mD_{r_1r_1} \geq \mD_{r'_1r_1} \geq {\varepsilon \over R_1^2}$ holds from some $r_1,  r'_1\in[R_1]$, where $r_1\neq r'_1$. Then, for any $(r_2,\ldots,r_K)\in[d_2]\times \cdots \times [d_K]$ and any $(a_2,\ldots,a_K)\in[d_2]\times \cdots \times [d_K]$, the following inequality holds:
\begin{align}\label{eq:3}
&{ \tN (\mD^{(1)}, \ldots, \mD^{(K)})_{r_1 r_2\ldots r_K}\over w_{r_1r_2\ldots r_K}} -f (z_{r_1r_2 \ldots r_K})\nonumber \\
\geq & {1\over 2w_{r_1r_2\ldots r_K}} \left( \mD^{(1)}_{r_1r_1}\mD^{(2)}_{a_2r_2}\cdots  \mD^{(K)}_{a_Kr_K}(c_{r_1a_2\ldots a_K} -z_{r_1r_2\ldots r_K})^2+  \mD^{(1)}_{r'_1r_1}\mD^{(2)}_{a_2r_2}\cdots \mD^{(K)}_{a_Kr_K}(c_{r'_1a_2\ldots a_k} -z_{r_1r_2\ldots r_K})^2\right) \nonumber\\
\geq &{1\over 2w_{r_1r_2\ldots r_K}} \min\left\{\mD^{(1)}_{r_1r_1}, \mD^{(1)}_{r'_1r_1}\right\}   \left ( (c_{r_1a_2\ldots a_k} -z_{r_1r_2\ldots r_K})^2+(c_{r'_1a_2\ldots a_K} -z_{r_1r_2\ldots r_K})^2  \right)\mD^{(2)}_{a_2r_2}\cdots \mD^{(K)}_{a_Kr_K} \nonumber\\
%\geq &{1\over 4W}{\varepsilon\over R^2} \mD_{a_2r_2}\ldots \mD_{a_Kr_K} \left(c_{r_1a_2\ldots a_K}-c_{r'_1a_2\ldots,a_K}\right)^2\\
\geq & {1\over 4w_{r_1r_2\ldots r_K}}{\delta_{\min}\varepsilon \over R_1^2} \mD^{(2)}_{a_2r_2}\cdots \mD^{(K)}_{a_Kr_K}.
\end{align}
Here $\tN=\entry{f(c_{r_1 \ldots r_K})}\in\mathbb{R}^{R_1\times \cdots R_K}$ is the loss function evaluated at each block, $\tN (\mD^{(1)}, \ldots, \mD^{(K)})=\tN\times_1 {\mD^{(1)}}^T\times_2\cdots\times_K {\mD^{(K)}}^T$ is the weighted value of the loss function, $z_{r_1\ldots,r_K}={1\over w_{r_1\ldots r_K}}\tC(\mD^{(1)},\ldots,\mD^{(K)})_{r_1\ldots r_K}$ is the $(r_1\ldots r_k)$-th weighted entry of the block means.

Third: Taking sum of~\eqref{eq:3} over $(r_2,\ldots,r_K)$ gives
\begin{align}\label{eq:1}
\sum_{r_2,\ldots,r_K}\left( w_{r_1 r_2 \ldots r_K}f(z_{r_1r_2\ldots r_K})-\tN(\mD^{(1)},\ldots,\mD^{(K)})_{r_1 r_2\ldots r_K} \right)& \leq - {1\over 4} {\delta_{\min}\varepsilon \over R^2_1} \sum_{r_2,\ldots,r_K}\mD^{(2)}_{a_2r_2}\cdots\mD^{(K)}_{a_Kr_K}\nonumber\\
&\leq -{\delta_{\min} \tau^{K-1}\varepsilon\over 4 R^2_1}.
\end{align}

Note that the inequality~\eqref{eq:1} holds for a certain $r_1\in[R_1]$. For any other $a_1=\{1,\ldots, R_1\}/\{r_1\}$, by Jensen's inequality we have
\begin{equation}\label{eq:2}
\sum_{a_2,\ldots,a_K}\left(w_{a_1a_2 \ldots a_K}f(z_{a_1a_2\ldots a_K})-\tN(\mD^{(1)},\ldots,\mD^{(K)})_{a_1a_2\ldots a_K}\right)\leq 0,\quad \text{for all }a_1\in [d_1]/\{r_1\}.
\end{equation}
Combining~\eqref{eq:1} and~\eqref{eq:2} yields
\[
\sum_{a_1,\ldots,a_K}...&=\sum_{a_1=r_1, (a_2,\ldots,a_K)\in[d_2]\times \cdots \times [d_K]}...+\sum_{a_1\in[R_1]/\{r_1\}, (a_2,\ldots,a_K)\in[d_2]\times \cdots \times [d_K]} ...\leq-{\delta_{\min}\tau^{K-1}\varepsilon\over 4R^2_1}.
\]
\end{document}