\documentclass[twoside]{article}

\usepackage{aistats2020}


\usepackage{graphicx}
\usepackage[utf8]{inputenc} % allow utf-8 input
\usepackage[T1]{fontenc}    % use 8-bit T1 fonts
\usepackage{hyperref}       % hyperlinks
\usepackage{url}            % simple URL typesetting
\usepackage{booktabs}       % professional-quality tables
\usepackage{amsmath,amssymb} 
\usepackage{amsthm}    % blackboard math symbols
\usepackage{nicefrac}       % compact symbols for 1/2, etc.
\usepackage{microtype}      % microtypography
\usepackage{bm}
\usepackage{subfig}
\usepackage[english]{babel}
\usepackage{algorithm}
%\usepackage{algorithmic}
\usepackage{appendix}

\theoremstyle{plain}
\newtheorem{thm}{Theorem}[section]
\newtheorem{lem}{Lemma}
\newtheorem{prop}{Proposition}
\newtheorem{pro}{Property}
\newtheorem{cor}{Corollary}
\newtheorem{ass}{Assumption}

\theoremstyle{definition}
\newtheorem{defn}{Definition}
\newtheorem{example}{Example}
\newtheorem{rmk}{Remark}

\input macros.tex
\usepackage{dsfont}

\usepackage{multirow}
\usepackage{algpseudocode,algorithm}
\algnewcommand\algorithmicinput{\textbf{Input:}}
\algnewcommand\algorithmicoutput{\textbf{Output:}}
\algnewcommand\INPUT{\item[\algorithmicinput]}
\algnewcommand\OUTPUT{\item[\algorithmicoutput]}

\DeclareMathOperator*{\minimize}{minimize}



\usepackage{mathtools}
\mathtoolsset{showonlyrefs}
\newcommand*{\KeepStyleUnderBrace}[1]{%f
  \mathop{%
    \mathchoice
    {\underbrace{\displaystyle#1}}%
    {\underbrace{\textstyle#1}}%
    {\underbrace{\scriptstyle#1}}%
    {\underbrace{\scriptscriptstyle#1}}%
  }\limits
}
\usepackage{xr}
\externaldocument{tensor_regression_supp}

% If your paper is accepted, change the options for the package
% aistats2020 as follows:
%
% \usepackage[accepted]{aistats2020}
%
% This option will print headings for the title of your paper and
% headings for the authors names, plus a copyright note at the end of
% the first column of the first page.

% If you set papersize explicitly, activate the following three lines:
%\special{papersize = 8.5in, 11in}
%\setlength{\pdfpageheight}{11in}
%\setlength{\pdfpagewidth}{8.5in}

% If you use natbib package, activate the following three lines:
%\usepackage[round]{natbib}
%\renewcommand{\bibname}{References}
%\renewcommand{\bibsection}{\subsubsection*{\bibname}}

% If you use BibTeX in apalike style, activate the following line:
%\bibliographystyle{apalike}

\begin{document}

% If your paper is accepted and the title of your paper is very long,
% the style will print as headings an error message. Use the following
% command to supply a shorter title of your paper so that it can be
% used as headings.
%
%\runningtitle{I use this title instead because the last one was very long}

% If your paper is accepted and the number of authors is large, the
% style will print as headings an error message. Use the following
% command to supply a shorter version of the authors names so that
% they can be used as headings (for example, use only the surnames)
%
%\runningauthor{Surname 1, Surname 2, Surname 3, ...., Surname n}

\twocolumn[

\aistatstitle{Generalized tensor response regression with multilinear covariates}

\aistatsauthor{ Anonymous Author 1 \And Anonymous Author 2 \And Anonymous Author 3}

\aistatsaddress{ Unknown Institution 1 \And  Unknown Institution 2 \And Unknown Institution 3 } ]

\begin{abstract}
We consider the problem of learning higher-order tensor with side information on a set of modes. Such data problems arise frequently arise in applications such as neuroimaging, network analysis, and ... We propose a new family of tensor response regression models that incorporate covariate information. 
\end{abstract}

\section{Introduction}

Many contemporary scientific and engineering studies collect multi-way array data, a.k.a.\ tensor, accompanied by additional covariates. For example, in neuro-imaging analysis, researchers measure brain connections from a sample of individuals with the goal to identifying the brain edges affected by individual covariates. In social network analysis, explain the connection (comminitity) by attributable of both nodes. ... 
(add two pictures; one for estimating network population; another for estimating link prediction)
In this article, we provide a general treatment to these seemingly different problems.

\section{General Model}
Let $\tY=\entry{y_{i_1,\ldots,i_K}}\in\mathbb{R}^{d_1\times \cdots\times d_K}$ denote an order-$K$ data tensor of interest. In addition, suppose we observe covariate $\mX_k=\entry{x^{(k)}_{pi}}\in\mathbb{R}^{p_k\times d_k}$ on the mode-$k$, where $x^{(k)}_{pi}$ is the $p$-th covariate of entry $i$ along the mode $k$. We propose the following multilinear structure in the mean of the tensor. Specifically, 
\begin{align}
&\mathbb{E}(\tY|\mX_1,\ldots,\mX_K)=f(\Theta),\ \text{where}\\
&\Theta =\tB\times\{\mX_1,\ldots,\mX_K\} ,
\end{align}
where $f(\cdot)$ is a known link function, $\Theta\in\mathbb{R}^{d_1\times \cdots\times d_K}$ is the linear predictor, $\tB\in\mathbb{R}^{p_1\times \cdots p_K}$ is the parameter tensor of interest, and $\mX_k\in\mathbb{R}^{d_k\times p_k}$ are known covariate matrices, and $\times$ denotes the tensor Tucker product. We give three examples of multi-covariates tensor regression model arises in literature. 

\begin{example}[Spatio-temporal growth model]
Let $\tY=\entry{y_{ijk}}\in\mathbb{R}^{d \times m\times n}$ denote the pH measurements of $d$ lakes at $m$ levels of depth and for $n$ time points. Suppose the sampled lakes belong to $q$ types, with $p$ lakes in each type. Let $\{\ell_j\}_{j\in[m]}$ denote the depth levels and $\{t_k\}_{k\in[n]}$ the time points. Assume the expected pH trend in depth is a polynomial of order $r-1$ and that the expected trend in time is a polynomial of order $s-1$. Then, a classical spatio-temporal growth model can be represented as
\[
\mathbb{E}(\tY|\mX_1,\mX_2,\mX_3)=\tB\times\{\mX_1,\mX_2,\mX_3\},
\]
where $\tB\in\mathbb{R}^{p\times r\times s}$ is the coefficient tensor of interest, $\mX_1\in \{0,1\}^{d\times p}$ is the design matrix for lake types, 
\[
\mX_2=
\begin{psmallmatrix}
1 & \ell_1&\cdots &\ell^{r-1}_1\\
1 & \ell_2&\cdots &\ell^{r-1}_2\\
\vdots &\vdots&\ddots&\vdots\\
1&\ell_{m}&\cdots&\ell^{r-1}_{m}
\end{psmallmatrix},\quad
\mX_3=
\begin{psmallmatrix}
1 & t_1&\cdots &t^{s-1}_1\\
1 & t_2&\cdots &t^{s-1}_2\\
\vdots &\vdots&\ddots&\vdots\\
1&t_{n}&\cdots&t^{s-1}_{n}
\end{psmallmatrix}
\]
are the design matrices for spatial and temporal effects, respectively. 
\end{example}
\begin{example}[Network population model] 
Network response model is a very recent model in neuroimanig. The model studies the relationship between the network-valued response with the individual covariates. Suppose we observe $n$ i.i.d.\ observation $\{(\mY_i, \mx_i): i=1,\ldots,n\}$, where $\mY_i\in\{0,1\}^{d\times n}$ is the brain connectivity network on the $i$-th individual and $\mx_i\in\mathbb{R}^p$ is the subject covariate such as age, gender. The network-response model is of the form
\begin{equation}\label{eq:network}
\text{logit}(\mathbb{E}(\mY_i|\mx_i))=\tB\times_3\mx_i, \quad \text{for }i=1,\ldots,n
\end{equation}
where $\tB\in \mathbb{R}^{d\times d\times p}$ is the coefficient tensor of interest. In fact, the model~\eqref{eq:network} is a special case of our multilinear tensor-response model. To see this, let $\tY\in\{0,1\}^{d\times d\times n}$ denote the response tensor by stacking $\{\mY_i\}$ together along the 3$^\text{rd}$ mode and $\mX=[\mx_1,\ldots,\mx_n]\in\mathbb{R}^{p\times n}$, then model~\eqref{eq:network} can be expressed as 
\[
\text{logit}(\mathbb{E}(\tY|\mX))=\tB\times_3 \mX=\tB\times\{\mI_d, \mI_d, \mX\},
\]
where $\mI_d$ denotes the identity matrix of dimension $d$. 
 \end{example}
 
 \begin{example}[Link model with node attributes] Let $V=[n]$ be a set of vertices and explanatory variable $x_i\in\mathbb{R}^p$ associated to each $i\in V$. The network $G=(V,E)$ is described by the following matrix model. The edge connects the two vertices $i$ and $j$ independently of the others is modeled as
 \[
 \text{logit}(\mathbb{P}((i,j)\in E)=\mx^T_i\mB\mx_j=\langle \mB, \mx^T_i\mx_j\rangle.
 \]
 Let $\tY=\entry{y_{ij}}, wehre y_{ij}=\mathds{1}_{(i,j)\in E}$. Define $\mX=[\mx_1,\ldots,\mx_n]\in\mathbb{R}^{p\times n}$. Then the above model can be expressed as
 \[
 \text{logit}(\mathbb{E}(\mY))=\mB\times_1\mX\times_2\mX
 \]
\end{example}
In the above three example and many other studies, researchers are interested in identifying the regions of tensor that are associated to the covariates.
\subsection{Second Level Heading}

Second level headings are initial caps, flush left, bold, and in point
size 10. Use one line space before the second level heading and one-half line
space after the second level heading.

\subsubsection{Third Level Heading}

Third level headings are flush left, initial caps, bold, and in point
size 10. Use one line space before the third level heading and one-half line
space after the third level heading.

\paragraph{Fourth Level Heading}

Fourth level headings must be flush left, initial caps, bold, and
Roman type.  Use one line space before the fourth level heading, and
place the section text immediately after the heading with no line
break, but an 11 point horizontal space.

\subsection{CITATIONS, FIGURES, REFERENCES}


\subsubsection{Citations in Text}

Citations within the text should include the author's last name and
year, e.g., (Cheesman, 1985). References should follow any style that
you are used to using, as long as their style is consistent throughout
the paper.  Be sure that the sentence reads correctly if the citation
is deleted: e.g., instead of ``As described by (Cheesman, 1985), we
first frobulate the widgets,'' write ``As described by Cheesman
(1985), we first frobulate the widgets.''  %Be sure to avoid
%accidentally disclosing author identities through citations.

\subsubsection{Footnotes}

Indicate footnotes with a number\footnote{Sample of the first
  footnote.} in the text. Use 8 point type for footnotes. Place the
footnotes at the bottom of the column in which their markers appear,
continuing to the next column if required. Precede the footnote
section of a column with a 0.5 point horizontal rule 1~inch (6~picas)
long.\footnote{Sample of the second footnote.}

\subsubsection{Figures}

All artwork must be centered, neat, clean, and legible.  All lines
should be very dark for purposes of reproduction, and art work should
not be hand-drawn.  Figures may appear at the top of a column, at the
top of a page spanning multiple columns, inline within a column, or
with text wrapped around them, but the figure number and caption
always appear immediately below the figure.  Leave 2 line spaces
between the figure and the caption. The figure caption is initial caps
and each figure should be numbered consecutively.

Make sure that the figure caption does not get separated from the
figure. Leave extra white space at the bottom of the page rather than
splitting the figure and figure caption.
\begin{figure}[h]
\vspace{.3in}
\centerline{\fbox{This figure intentionally left non-blank}}
\vspace{.3in}
\caption{Sample Figure Caption}
\end{figure}

\subsubsection{Tables}

All tables must be centered, neat, clean, and legible. Do not use hand-drawn tables.
Table number and title always appear above the table.
See Table~\ref{sample-table}.

Use one line space before the table title, one line space after the table title,
and one line space after the table. The table title must be
initial caps and each table numbered consecutively.

\begin{table}[h]
\caption{Sample Table Title} \label{sample-table}
\begin{center}
\begin{tabular}{ll}
\textbf{PART}  &\textbf{DESCRIPTION} \\
\hline \\
Dendrite         &Input terminal \\
Axon             &Output terminal \\
Soma             &Cell body (contains cell nucleus) \\
\end{tabular}
\end{center}
\end{table}

\section{SUPPLEMENTARY MATERIAL}

If you need to include additional appendices during submission, you
can include them in the supplementary material file.



\section{INSTRUCTIONS FOR CAMERA-READY PAPERS}

For the camera-ready paper, if you are using \LaTeX, please make sure
that you follow these instructions.  (If you are not using \LaTeX,
please make sure to achieve the same effect using your chosen
typesetting package.)

\begin{enumerate}
    \item Download \texttt{fancyhdr.sty} -- the
    \texttt{aistats2020.sty} file will make use of it.
    \item Begin your document with
    \begin{flushleft}
    \texttt{\textbackslash documentclass[twoside]\{article\}}\\
    \texttt{\textbackslash usepackage[accepted]\{aistats2020\}}
    \end{flushleft}
    The \texttt{twoside} option for the class article allows the
    package \texttt{fancyhdr.sty} to include headings for even and odd
    numbered pages. The option \texttt{accepted} for the package
    \texttt{aistats2020.sty} will write a copyright notice at the end of
    the first column of the first page. This option will also print
    headings for the paper.  For the \emph{even} pages, the title of
    the paper will be used as heading and for \emph{odd} pages the
    author names will be used as heading.  If the title of the paper
    is too long or the number of authors is too large, the style will
    print a warning message as heading. If this happens additional
    commands can be used to place as headings shorter versions of the
    title and the author names. This is explained in the next point.
    \item  If you get warning messages as described above, then
    immediately after $\texttt{\textbackslash
    begin\{document\}}$, write
    \begin{flushleft}
    \texttt{\textbackslash runningtitle\{Provide here an alternative
    shorter version of the title of your paper\}}\\
    \texttt{\textbackslash runningauthor\{Provide here the surnames of
    the authors of your paper, all separated by commas\}}
    \end{flushleft}
    Note that the text that appears as argument in \texttt{\textbackslash
      runningtitle} will be printed as a heading in the \emph{even}
    pages. The text that appears as argument in \texttt{\textbackslash
      runningauthor} will be printed as a heading in the \emph{odd}
    pages.  If even the author surnames do not fit, it is acceptable
    to give a subset of author names followed by ``et al.''

    \item Use the file sample\_paper.tex as an example.

    \item The camera-ready versions of the accepted papers are 8
      pages, plus any additional pages needed for references.

    \item If you need to include additional appendices,
      you can include them in the supplementary
      material file.

    \item Please, don't change the layout given by the above
      instructions and by the style file.

\end{enumerate}

\subsubsection*{Acknowledgements}

Use the unnumbered third level heading for the acknowledgements.  All
acknowledgements go at the end of the paper.

\subsubsection*{References}

References follow the acknowledgements.  Use an unnumbered third level
heading for the references section.  Any choice of citation style is
acceptable as long as you are consistent.  Please use the same font
size for references as for the body of the paper---remember that
references do not count against your page length total.

\begin{thebibliography}{}
\setlength{\itemindent}{-\leftmargin}
\makeatletter\renewcommand{\@biblabel}[1]{}\makeatother
\bibitem{} J.~Alspector, B.~Gupta, and R.~B.~Allen (1989).
    \newblock Performance of a stochastic learning microchip.
    \newblock In D. S. Touretzky (ed.),
    \textit{Advances in Neural Information Processing Systems 1}, 748--760.
    San Mateo, Calif.: Morgan Kaufmann.

\bibitem{} F.~Rosenblatt (1962).
    \newblock \textit{Principles of Neurodynamics.}
    \newblock Washington, D.C.: Spartan Books.

\bibitem{} G.~Tesauro (1989).
    \newblock Neurogammon wins computer Olympiad.
    \newblock \textit{Neural Computation} \textbf{1}(3):321--323.
\end{thebibliography}
\end{document}
