\documentclass[11pt]{article}

\usepackage{fancybox}



\usepackage{color}
\usepackage{url}
\usepackage[margin=1in]{geometry}


\renewcommand{\textfraction}{0.0}
\renewcommand{\topfraction}{1.0}
%\renewcommand{\textfloatsep}{5mm}




\usepackage{comment}
% Definitions of handy macros can go here
\usepackage{amsmath,amssymb,amsthm,bm,mathtools}
%\usepackage{dsfont,multirow,hyperref,setspace,natbib,enumerate}
\usepackage{dsfont,multirow,hyperref,setspace,enumerate}
\hypersetup{colorlinks,linkcolor={blue},citecolor={blue},urlcolor={red}} 
\usepackage{algpseudocode,algorithm}
\algnewcommand\algorithmicinput{\textbf{Input:}}
\algnewcommand\algorithmicoutput{\textbf{Output:}}
\algnewcommand\INPUT{\item[\algorithmicinput]}
\algnewcommand\OUTPUT{\item[\algorithmicoutput]}

\mathtoolsset{showonlyrefs=true}



\theoremstyle{plain}
\newtheorem{thm}{Theorem}[section]
\newtheorem{lem}{Lemma}
\newtheorem{prop}{Proposition}
\newtheorem{pro}{Property}
\newtheorem{assumption}{Assumption}

\theoremstyle{definition}
\newtheorem{defn}{Definition}
\newtheorem{cor}{Corollary}
\newtheorem{example}{Example}
\newtheorem{rmk}{Remark}


\def\MLET{\hat \Theta_{\text{MLE}}}
\newcommand{\cmt}[1]{{\leavevmode\color{red}{#1}}}



\usepackage{dsfont}

\usepackage{multirow}

\DeclareMathOperator*{\minimize}{minimize}



\usepackage{mathtools}
\mathtoolsset{showonlyrefs}
\newcommand*{\KeepStyleUnderBrace}[1]{%f
  \mathop{%
    \mathchoice
    {\underbrace{\displaystyle#1}}%
    {\underbrace{\textstyle#1}}%
    {\underbrace{\scriptstyle#1}}%
    {\underbrace{\scriptscriptstyle#1}}%
  }\limits
}
\usepackage{xr}
\externaldocument{tensor_regression}
\input macros.tex





\title{Supplements for ``Generalized tensor response regression with multi-sided covariates''}


%\author{%
%Yuchen Zeng \\
%University of Wisconsin -- Madison\\
 %\texttt{yzeng58@wisc.edu} \\
%\And
%Miaoyan Wang \\
%University of Wisconsin -- Madison\\
%\texttt{miaoyan.wang@wisc.edu} \\
%}

\begin{document}


\begin{center}
\begin{spacing}{1.5}
\textbf{\Large Supplements for ``Generalized tensor-response model with multi-sided covariates''}
\end{spacing}
\end{center}

\section{Proofs}
\begin{thm}
Consider a generalized tensor regression model with multi-sided covariates $\tX=\{\mX_1,\ldots,\mX_K\}$. Suppose the entries in $\tY$ are independent realizations of an exponential family distribution, and $\mathbb{E}(\tY|\tX)$ follows the low-rank tensor regression model~\eqref{eq:tensormodel}. Under Assumption~\ref{ass}, there exist two absolute constants $C_1, C_2>0$, such that, with probability at least $1-\exp(-C_1\sum_k p_k)$, 
\begin{equation}\label{eq:bound}
\text{Loss}(\trueB,\ \hat \tB) \leq C_3\sum_k p_k,
\end{equation}
where $C_3=C_3(\mr)={1\over C^{2K}_2U}{\prod_k r_k \over \max_k r_k}>0$ is a constant that does not depend on the dimensions $\{d_k\}$ and $\{p_k\}$. 
\end{thm}


\begin{proof}[Proof of Theorem 1]

Let $\ell(\tB)=\mathbb{E}(\tL_{\tY}(\tB))$, where the expectation is take with respect to $\tY\sim \trueB$ under the true parameter. We show that 
\begin{enumerate}
\item[C1.] The stochastic deviation $\tL_{\tY}(\tB)-\ell(\tB)$ is uniformly small for all $\tB\in\tP$,
\item[C2.] There exist two positive constants $c_1, c_2>0$ such that 
\[
c_1 \FnormSize{}{\hat \tB-\trueB}^2 \leq \ell(\hat \tB) - \ell(\trueB) \leq  c_2\FnormSize{}{\hat \tB-\trueB}^2.
\]
\end{enumerate}

To prove C1, note that 
\begin{align}
\tL_{\tY}(\tB)-\ell(\tB)&=\langle \tY-\mathbb{E}(\tY|\tX),\ \Theta(\tB)\rangle\\
&= \langle \tY- b'(\trueT),\ \Theta\rangle \\
&= \langle \tE\times_1\mX^T_1\times_2\cdots\times_K\mX^T_K,\ \tB\rangle,
\end{align}
where $\tE=\entry{\varepsilon_{i_1,\ldots,i_K}}\stackrel{\text{def}}{=}\tY-b'(\trueT)$. Based on Assumption A1, $\tE$ is a sub-Gaussian tensor with parameter bounded by $C_1=\phi U$. Therefore, $\check\tE\stackrel{\text{def}}{=}\tE\times_1\mX^T_1\times_2\cdots\times_K\mX^T_K$ is a $(p_1,\ldots,p_K)$-dimensional sub-Gaussian with parameter bounded by $C_2=\phi Uc^{2K}_2$. By Cauchy-Schwarz inequality,
\[
|\tL_{\tY}(\tB)-\ell(\tB)|\leq \norm{\check \tE} \nnorm{\tB}.
\]
where $\norm{\cdot}$ denotes the tensor spectral norm and $\nnorm{\cdot}$ denotes the tensor nuclear norm. 

We have that $\nnorm{\tB}\leq {\prod_k r_k \over \max_k r_k}\FnormSize{}{\tB}$ by~\cite{wang2018learning,wang2017operator}.
Moreover, the Gaussian tensor theory~\cite{tomioka2014spectral} shows that $\norm{\check \tE}\leq C_1\sum_k p_k$ with probability at least $1-\exp(-C_2\sum_kp_k)$

To prove C2, we note that 
\begin{equation}\label{eq:log}
\ell(\tB)=\ell(\trueB)-{1\over 2}\text{vec}(\tB-\trueB)^T\mathbb{E}(\tH_{\tY}(\check \tB))\text{vec}(\tB-\trueB),
\end{equation}
where $\tH_{\tY}(\check \tB)$ is the Hession of ${\partial \ell^2 (\tB)\over\partial^2 \tB}$ evaluated at $\check \tB =\alpha \text{vec}(\alpha \tB+(1-\alpha)\trueB)$ for some $\alpha\in[0,1]$. Recall that $b''(\theta)=\text{Var}(y|\theta)$ if $y\in\mathbb{R}$ follows the exponential family distribution with function $b(\cdot)$. Therefore, the equation~\eqref{eq:log} can be written as
\[
\ell(\tB)-\ell(\trueB)=-{1\over 2}\sum_{i_1,\ldots,i_K}b''(\check \theta_{i_1,\ldots,i_K}) (\theta_{i_1,\ldots,i_K}-\theta_{\text{true},i_1,\ldots,i_K})^2 \leq -{L \over 2}\FnormSize{}{\Theta-\trueT}^2,
\]
holds for all $\tB\in\tP$, provided that $\min_{|\theta|\leq \alpha}|b''(\theta)|\geq L>0$.

Now we consider the constrained MLE $\hat \tB$. By definition, $\tL_{\tY}(\hat \tB)- \tL_{\tY}(\trueB)\geq 0$. This implies that
\begin{align}
0&\leq \tL_{\tY}(\hat \tB)- \tL_{\tY}(\trueB) \\
&\leq \left(\tL_{\tY}(\hat \tB)-\ell(\hat \tB)\right)-\left( \tL_{\tY}(\trueB)-\ell(\trueB)\right)+\left(\ell(\hat \tB)-\ell(\trueB)\right)\\
&\leq 2\sup_{\tB\in\tP}|\tY| -{L\over 2}\FnormSize{}{\hat \Theta-\trueT}^2\\
&\leq 2\sup_{\tB}|\tL_{\tY}(\tB)-\ell(\tB)|-{L\over 2}\FnormSize{}{\hat \Theta-\trueT}^2
\end{align}
Therefore, the statement
\begin{align}\label{eq:1}
\FnormSize{}{\hat \Theta-\trueT}&\leq {2\over L}\big\langle \tE,\ {\hat \Theta -\trueT \over \FnormSize{}{\hat \Theta-\trueT}} \big\rangle\\
&\leq {2\over L}\sup_{\Theta: \FnormSize{}{\Theta}=1, \Theta=\tB\times_1\mX_1\times_2\cdots\times_K \mX_K}\langle \tE,\ \Theta \rangle\\
&\leq {2\over L}\sup_{\tB\in\tP: \FnormSize{}{\tB}\leq \prod_k \sigma^{-1}_{\min}(\mX_k)} \langle \tE\times_1\mX^T_1\times_2\cdots\times_K \mX^T_K,\ \tB\rangle.
\end{align}
Combining~\eqref{eq:1} with C1 yields the final conclusion. 

%\[
%\FnormSize{}{\hat \Theta-\trueT}\leq {2R\over L}  \sum_{i=1} p_i ()
%\]
%\[
%\FnormSize{}{\hat \Theta-\trueT}^2\leq {2\over L}\sup_{\tB: \FnormSize{}{\tB}=1}|\tL_{\tY}(\tB)-\ell(\tB)|\leq {2\over L} \sum_k p_k
%\]
%holds with probability at least $1-\exp(C_1\sum_k p_k)$.
\end{proof}

\section{Real data analysis}
\begin{enumerate}
\item  commonbloc0, blockpositionindex
\item 
officialvisits, violentactions, militaryactions, duration, negativebehavior, boycottembargo, aidenemy, negativecomm, accusation"         "protests"           "unoffialacts"      
[13] "nonviolentbehavior" "emigrants"          "relexports"        
[16] "timesincewar"       "commonbloc2"         "intergovorgs3" "relintergovorgs""intergovorgs" 
\item economicaid"         "releconomicaid"      "conferences"        
 [4] "booktranslations"    "relbooktranslations" "severdiplomatic"    
 [7] "expeldiplomats"      "attackembassy"       "unweightedunvote"   
[10] "tourism"             "reltourism"          "tourism3"           
[13] "relemigrants"        "emigrants3"          "students"           
[16] "relstudents"         "exports"             "exports3"           
[19] "lostterritory"       "dependent"           "militaryalliance"   "warning"    
\item 
 [1] "treaties"        "reltreaties"     "exportbooks"     "relexportbooks" 
 [5] "weightedunvote"    "ngo"            "relngo"          "ngoorgs3"        "embassy"        
[13] "reldiplomacy"    "timesinceally"   "independence"    "commonbloc1"   
\end{enumerate}


\bibliographystyle{unsrt}
\bibliography{tensor_wang}

\end{document}