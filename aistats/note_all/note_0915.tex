\documentclass[11pt]{article}
\usepackage{lscape}
\usepackage{amsmath,amssymb}
\usepackage{amsthm}
\usepackage{float}
\usepackage{booktabs}
\usepackage{graphicx}
\usepackage{comment}
\usepackage{bm}
\usepackage{gensymb}
\allowdisplaybreaks[4]
\usepackage{geometry}
\geometry{margin=1in}
\usepackage{setspace}
\usepackage{siunitx}
\usepackage{enumitem}
\usepackage{dsfont}

\usepackage{graphics}
\usepackage[noend]{algorithmic}
\usepackage[utf8x]{inputenc}
\usepackage{bm}

\usepackage{hyperref}
\hypersetup{
    colorlinks=true,
    citecolor = blue,
    linkcolor=blue,
    filecolor=magenta,           
    urlcolor=cyan,
}


\theoremstyle{plain}
\newtheorem{thm}{Theorem}
\newtheorem{lem}{Lemma}
\newtheorem{prop}{Proposition}
\newtheorem{pro}{Property}
\newtheorem{cor}{Corollary}
\newtheorem{ass}{Assumption}

\theoremstyle{definition}
\newtheorem{defn}{Definition}
\newtheorem{exmp}{Example}
\newtheorem{rmk}{Remark}



\usepackage[labelfont=bf]{caption}

\setcounter{table}{1}
 % \usepackage[labelformat=empty]{ caption}
\usepackage{multirow}
\usepackage{tabularx}

\def\fixme#1#2{\textbf{[FIXME (#1): #2]}}

 

\newcommand*{\KeepStyleUnderBrace}[1]{%f
  \mathop{%
    \mathchoice
    {\underbrace{\displaystyle#1}}%
    {\underbrace{\textstyle#1}}%
    {\underbrace{\scriptstyle#1}}%
    {\underbrace{\scriptscriptstyle#1}}%
  }\limits
}
% \usepackage[initials]{amsrefs}
%\usepackage{amsaddr}
\usepackage{mathtools}
\mathtoolsset{showonlyrefs=true}


\usepackage{hyperref}
\hypersetup{colorlinks=true}
\usepackage[parfill]{parskip}
\usepackage{bm}
\onehalfspacing
%%%%%%%%%%%%%%%%%%%%%%%%%%%%%%%%%%%%%%%%%%%%%%%%%%%%%%%%%%%%%%%%%%%%%
%%             Math Symbols
%%%%%%%%%%%%%%%%%%%%%%%%%%%%%%%%%%%%%%%%%%%%%%%%%%%%%%%%%%%%%%%%%%%%%

%%               Bold Math
\input macros.tex
\def\refer#1{\emph{\color{blue}#1}}

\usepackage{sectsty}
\sectionfont{\fontsize{12}{15}\selectfont}

\newcommand*{\QEDB}{\hfill\ensuremath{\square}}%


%\author{%
%Yuchen Zeng \\
%University of Wisconsin -- Madison\\
 %\texttt{yzeng58@wisc.edu} \\
%\And
%Miaoyan Wang \\
%University of Wisconsin -- Madison\\
%\texttt{miaoyan.wang@wisc.edu} \\
%}

\begin{document}
\begin{center}
{\bf \large Generalized tensor response regression with multilinear features
\end{center}

\section{Preliminaries}
Let $\tY=\entry{y_{i_1,\ldots,i_K}}\in \mathbb{R}^{d_1\times \cdots \times d_K}$ be an order-$K$ tensor, and $\mX_k \in \mathbb{R}^{d_k\times p_k}$ be the features along the $k$-th mode.  
\begin{align}
f(\mathbb{E}(\tY|\mX_1,\ldots,\mX_K))=\tB\times_1\mX_1\times_2\cdots \times_K \mX_K
\end{align}
where $f(\cdot)$ is the inverse link function which is assumed to be applied to tensors in an entry-wise manner. Here $\tB\in\mathbb{R}^{p_1\times \cdots p_K}$ is the unknown parameter tensor of interest. 

\begin{align}
\tY |\mX_1,\ldots,\mX_K \stackrel{\text{in D.}}{\sim}& \exp\left(\langle \tY, \tU \rangle - \sum_{i_1,\ldots,i_K}b(u_{i_1,\ldots,i_K})\right)  \prod_{i_1,\ldots,i_K}h(y_{i_1,\ldots,i_K})\\
\text{where } \tU=\tB\times_1\mX_1\times_2\cdots\times_K \mX_K
\end{align}
where $b'(\cdot)$ is a link function and $\tU=\entry{u_{i_1,\ldots,i_K}}$ is the canonical parameter 
\[
\mathbb{E}(y_{i_1,\ldots,i_K}|\mX_1,\ldots,\mX_K)=b'(u_{i_1,\ldots,i_K})=\tB\times_1\mX_1\times_2 \cdots \times_K \mX_K
\]

Suppose that the response tensor $\tY=\entry{Y_{i_1,\ldots,i_K}}$ follows an exponential family distribution:
\[
f(\tY|\tX, \Theta)=\prod_{i_1,\ldots,i_K} f(y_{i_1,\ldots,i_K}|\tX,\Theta)=\prod_{i_1,\ldots,i_K} c(y_{i_1,\ldots,i_K}) \exp\left( y_{i_1,\ldots,i_K} \theta_{i_1,\ldots,i_K} - b(\theta_{i_1,\ldots,i_K}) \over \phi \right).
\]
Here $\phi$ is a constant, $c(\cdot)$ and $b(\cdot)$ are known functions, and $\Theta=\entry{\theta_{i_1,\dots,i_K}}$ collects the linear predictors and is modeled as
\[
\Theta=\tB\times_1\mX_1\times_2\cdots\times_K\mX_K,\quad\text{rank}(\Theta)\leq (r_1,\ldots,r_K),
\]
where $\tB\in\mathbb{R}^{p_1\times \cdots \times p_K}$ is the unknown coefficient tensor of interest. 
\begin{table}[H]
\centering
\begin{tabular}{cccc}
&$b(\theta)$&$b'(\theta)$&$c(\theta)&\phi\\
\hline
Normal & ${1\over 2}\theta^2$&$\theta$ & ...& $\sigma^2$\\
Bernoulli &$$ &${1\over 1+e^{-\theta}}$ & ...& 1\\
Poisson & $e^(\theta)$&$e^{\theta}$ & ...& 1\\
\end{tabular}
\end{table}

The negative log-likelihood for the coefficient tensor $\tB$ is
\[
\tL_\tY(\tB)=-\langle \tY, \Theta \rangle + \sum_{i_1,\ldots,i_K}b(\theta_{i_1,\ldots,i_K}),\quad \text{where}\quad \Theta=\Theta(\tB)=\tB\times_1\mX_1\times_2\cdots\times_K \mX_K.
\]

\end{document}