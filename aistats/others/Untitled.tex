\documentclass[10pt]{article}
\usepackage{setspace}
\usepackage{amsmath,amssymb}
\usepackage{amsthm}
\usepackage{fancybox}
\usepackage{algorithm, algpseudocode}
\usepackage{url}



\usepackage{enumitem}
\usepackage{multirow}
\usepackage{color}
\usepackage{graphicx}
\usepackage{setspace}
\usepackage{comment}
\usepackage{bm}

\begingroup
\renewcommand{\section}[2]{}%
%\renewcommand{\chapter}[2]{}% for othe

\newcommand*{\KeepStyleUnderBrace}[1]{%f
  \mathop{%
    \mathchoice
    {\underbrace{\displaystyle#1}}%
    {\underbrace{\textstyle#1}}%
    {\underbrace{\scriptstyle#1}}%
    {\underbrace{\scriptscriptstyle#1}}%
  }\limits
}


\usepackage[margin=1in]{geometry}
 
\allowdisplaybreaks[4]
\usepackage{bbm}


\usepackage{amsrefs}
\usepackage{mathtools}
\mathtoolsset{showonlyrefs=true}


\usepackage[utf8]{inputenc}
\usepackage{hyperref}
\hypersetup{
    colorlinks=true,
    citecolor = blue,
    linkcolor=blue,
    filecolor=magenta,           
    urlcolor=cyan,
}

\usepackage[breakable, theorems, skins]{tcolorbox}
\DeclareRobustCommand{\mybox}[2][gray!20]{%
\begin{tcolorbox}[   %% Adjust the following parameters at will.
        breakable,
        left=0pt,
        right=0pt,
        top=0pt,
        bottom=0pt,
        colback=#1,
        colframe=#1,
        width=\dimexpr\textwidth\relax, 
        enlarge left by=0mm,
        boxsep=5pt,
        arc=0pt,outer arc=0pt,
        ]
        #2
\end{tcolorbox}
}

\usepackage{algpseudocode,algorithm}
\algnewcommand\algorithmicinput{\textbf{Input:}}
\algnewcommand\algorithmicoutput{\textbf{Output:}}

\algnewcommand\INPUT{\item[\algorithmicinput]}
\algnewcommand\OUTPUT{\item[\algorithmicoutput]}

\input macros.tex
\begin{document}
\begin{center}
{\bf \large Comments on ``Tropp\_simulation.pdf''}\\
Miaoyan Wang, 09/28/2019
\end{center}
\begin{enumerate}
\item  
\end{enumerate}
\begin{algorithm}[H]
\caption{Approx tensor SVD 2}\label{alg:B}
\begin{algorithmic}[1]
\INPUT Tensor $\tA\in \mathbb{R}^{d_1\times \cdots \times d_K}$ and Tucker rank $(r_1,\ldots,r_K)$.
\OUTPUT Core tensor $\tS\in\mathbb{R}^{r_1\times \cdots \times r_K}$ and Tucker factors $\mQ^{(k)}\in\mathbb{R}^{d_k\times r_k}$.
\State {\bf Initialization.} Generate Gaussian test matrices $\mOmega_k$ of size $d_k\times r_k$, for all $k=1,\ldots,K$.
\For{$k$ in $\{1,2,\ldots,K\}$}
\State Form a wide matrix $\tA^{(k)}=\text{Unfold}_{k}(\tA\times_1 \mOmega^T_1\times_2 \cdots \times_{k-1}\mOmega^T_{k-1}\times_{k+1}\mOmega^T_{k+1}\times \cdots\times_K \mOmega^T_k)$, where $\text{Unfold}_k(\cdot)$ denotes the unfolding operation along the mode $k$. Note that the matrix $\tA^{(k)}$ is of dimension $d_k\times \prod_{i\neq k}r_i$.
\State Find a matrix $\mQ^{(k)}\in\mathbb{R}^{d_k\times r_k}$ whose columns form an orthogonal basis for the range of $\tA^{(k)}$. 
\EndFor
\State Return the core tensor $\tS=\tA\times_1(\mQ^{(1)})^T\times_2\cdots\times_K(\mQ^{(K)})^T$ and Tucker factors $\mQ^{(k)}$ for all $k=1,\ldots,K$.
\end{algorithmic}
\end{algorithm}

Other comments:

\end{document}