\documentclass[11pt]{article}
\usepackage[explicit]{titlesec}
\setlength{\parindent}{0pt}
\setlength{\parskip}{1em}
\usepackage{hyphenat}
\usepackage{ragged2e}
\RaggedRight

% These commands change the font. If you do not have Garamond on your computer, you will need to install it.

\usepackage[T1]{fontenc}
\usepackage{amsmath, amsthm}
\usepackage{graphicx}

% This adjusts the underline to be in keeping with word processors.
\usepackage{soul}
\setul{.6pt}{.4pt}


% The following sets margins to 1 in. on top and bottom and .75 in on left and right, and remove page numbers.
\usepackage{geometry}
\geometry{vmargin={1in,1in}, hmargin={.85in, .85in}}
\usepackage{fancyhdr}
\pagestyle{fancy}
\pagenumbering{gobble}
\renewcommand{\headrulewidth}{0.0pt}
\renewcommand{\footrulewidth}{0.0pt}

% These Commands create the label style for tables, figures and equations.
\usepackage[labelfont={footnotesize,bf} , textfont=footnotesize]{caption}
\captionsetup{labelformat=simple, labelsep=period}
\newcommand\num{\addtocounter{equation}{1}\tag{\theequation}}
\renewcommand{\theequation}{\arabic{equation}}
\makeatletter
%\renewcommand\tagform@[1]{\maketag@@@ {\ignorespaces {\footnotesize{\textbf{Equation}}} #1.\unskip \@@italiccorr }}
\makeatother
\setlength{\intextsep}{10pt}
\setlength{\abovecaptionskip}{2pt}
\setlength{\belowcaptionskip}{-10pt}

\renewcommand{\textfraction}{0.10}
\renewcommand{\topfraction}{0.85}
\renewcommand{\bottomfraction}{0.85}
\renewcommand{\floatpagefraction}{0.90}

% These commands set the paragraph and line spacing
\titleformat{\section}
  {\normalfont}{\thesection}{1em}{\MakeUppercase{\textbf{#1}}}
\titlespacing\section{0pt}{0pt}{-10pt}
\titleformat{\subsection}
  {\normalfont}{\thesubsection}{1em}{\textbf{\textit{#1}}}
\titlespacing\subsection{0pt}{0pt}{-8pt}
\renewcommand{\baselinestretch}{1.15}

\titleformat{\subsubsection}
  {\normalfont}{\thesubsubsection}{0.5em}{\textbf{#1}}
\titlespacing\subsubsection{0pt}{0pt}{-8pt}

% This designs the title display style for the maketitle command
\makeatletter
\newcommand\sixteen{\@setfontsize\sixteen{17pt}{6}}
\renewcommand{\maketitle}{\bgroup\setlength{\parindent}{0pt}
\begin{flushleft}
\sixteen\bfseries \@title
\medskip
\end{flushleft}
\textit{\@author}
\egroup}
\makeatother

% This styles the bibliography and citations.
%\usepackage[biblabel]{cite}
\usepackage[sort&compress]{natbib}
\setlength\bibindent{2em}
\makeatletter
\renewcommand\@biblabel[1]{\textbf{#1.}\hfill}
\makeatother
\renewcommand{\citenumfont}[1]{\textbf{#1}}
\bibpunct{}{}{,~}{s}{,}{,}
\setlength{\bibsep}{0pt plus 0.3ex}

%math
\newtheorem{theorem}{Theorem}
\newtheorem{assumption}{Assumption}
\newtheorem{defn}{Definition}
\usepackage{amsfonts}
\usepackage{amsmath}%
\usepackage{MnSymbol}%
\usepackage{wasysym}%
\usepackage{stmaryrd} 

\input macros.tex


%%% track the change 
 %%% %%% %%% %%% %%% %%% %%% %%% %%% %%% %%% %%% %%% %%% %%% %%% %%% %%% %%% %%%
%DIF PREAMBLE EXTENSION ADDED BY LATEXDIFF
%DIF UNDERLINE PREAMBLE %DIF PREAMBLE
\RequirePackage[normalem]{ulem} %DIF PREAMBLE
\RequirePackage{color}\definecolor{RED}{rgb}{1,0,0}\definecolor{BLUE}{rgb}{0,0,1} %DIF PREAMBLE
\providecommand{\DIFaddtex}[1]{{\protect\color{blue}\uwave{#1}}} %DIF PREAMBLE
\providecommand{\DIFdeltex}[1]{{\protect\color{red}\sout{#1}}}                      %DIF PREAMBLE
%DIF SAFE PREAMBLE %DIF PREAMBLE
\providecommand{\DIFaddbegin}{} %DIF PREAMBLE
\providecommand{\DIFaddend}{} %DIF PREAMBLE
\providecommand{\DIFdelbegin}{} %DIF PREAMBLE
\providecommand{\DIFdelend}{} %DIF PREAMBLE
%DIF FLOATSAFE PREAMBLE %DIF PREAMBLE
\providecommand{\DIFaddFL}[1]{\DIFadd{#1}} %DIF PREAMBLE
\providecommand{\DIFdelFL}[1]{\DIFdel{#1}} %DIF PREAMBLE
\providecommand{\DIFaddbeginFL}{} %DIF PREAMBLE
\providecommand{\DIFaddendFL}{} %DIF PREAMBLE
\providecommand{\DIFdelbeginFL}{} %DIF PREAMBLE
\providecommand{\DIFdelendFL}{} %DIF PREAMBLE
%DIF END PREAMBLE EXTENSION ADDED BY LATEXDIFF
%DIF PREAMBLE EXTENSION ADDED BY LATEXDIFF
%DIF HYPERREF PREAMBLE %DIF PREAMBLE
\providecommand{\DIFadd}[1]{{\DIFaddtex{#1}}} %DIF PREAMBLE
%\texorpdfstring
\providecommand{\DIFdel}[1]{{\DIFdeltex{#1}}} %DIF PREAMBLE
%DIF END PREAMBLE EXTENSION ADDED BY LATEXDIFF

%%%%%%%%%%%%%%%%%%%%%%%%%%%%%%%%%%%%%%%%%%%%%%%%%




%%%%%%%%%%%%%%%%%%%%%%%%%%%%%%%%%%%%%%%%%%%%%%%%%

% Authors: Add additional packages and new commands here.  
% Limit your use of new commands and special formatting.

% Place your title below. Use Title Capitalization.

\title{
\centering 
TR Global Convergence 0610}

% Add author information below. Communicating author is indicated by an asterisk, the affiliation is shown by superscripted lower case letter if several affiliations need to be noted.

\author{ \centering
Jiaxin Hu\\ \medskip 06/10/2020 \\ 
}

\pagestyle{empty}
\begin{document}

% Makes the title and author information appear.
\vspace*{.01 in}
\maketitle
\vspace{.12 in}

% Abstracts are required.
\section*{Global Convergence Property for TR Algorithm 1}

Here we study the global convergence property of iterates generated by Algorithm 1. For simplicity, let $\tA$ denote the decision variables $(\tC, \{M_k\})$.

\begin{theorem}[Global Convergence] 
	Assume the set $\{\tA|\ \tL(\tA) \geq \tL(\tA^{(0)}) \}$ is compact and the stationary points of $\tL(\tA)$ are isolated module equivalence. Then any sequence $\tA^{(t)}$ generated by alternating algorithm converges to a stationary point of $\tL(\tA)$ up to equivalence. 
\end{theorem}


\section*{Proof}

Pick an arbitrary iterate $\tA^{(t)}$.  Because of the compactness of set $\{\tA|\ \tL(\tA) \geq \tL(\tA^{(0)}) \}$, the domain of $\tA^{(t)}$ is bounded and thus there exists convergent sub-sequences of $\tA^{(t)}$.  Let $\tA^*$ denote a limiting points of $\tA^{(t)}$. Since $\tL(\tA^{(t)})$ increases monotonically along with $t \rightarrow \infty$, then $\tA^*$ is a stationary point of $\tL(\tA)$. 
 Let $\tS = \{\tA^*\}$ denote the  set of all the limiting points of $\tA^{(t)}$. We have $\tS \subset \{\tA|\ \tL(\tA) \geq \tL(\tA^{(0)}) \}$ and thus $\tS$ is a compact set. According to [Lange, 2012], $\tS$ is also connected. 
 
 
 Consider the equivalence of Tucker tensor representation. We define the equivalent class of $\tA$ as:
 \begin{align*}
 	\tE(\tA) = \{ \tA'|\ M'_k = M_kP^T_k, \tC' = \tC \times \{P_k\}, \text{ where } P^T_k \in \mathbb{O}_{r_k}, \forall k \in [K] \}.
 \end{align*}
  Notice that, for arbitrary $\tA$, $\tE(\tA)$ is a non-empty open set. For arbitrary two non-equivalent points $\tA_1$ and $\tA_2$, we have $\tE(\tA_1) \bigcap \tE(\tA_2) = \varnothing$ and thus $\tE(\tA_1) \bigcup \tE(\tA_2)$ is not connected. Using the definition of equivalent class, let $\tS_E$ denote the enlarged set of $\tS$, such that:
 \begin{align*}
 	\tS_E = \bigcup_{\tA \in \tS} \tE(\tA).
 \end{align*} 
The enlarged set $\tS_E$ satisfies below two properties:
 \begin{enumerate}
 \vspace{-.5cm}
 	\item [1.] [Union of Stationary Point] The set $\tS_E$ is an union of equivalent  classes generated by the stationary points in $\tS$.
 	\item [2.] [Connectedness model equivalence]  The set $\tS_E$ is connected between different equivalent classes.
\vspace{-.5cm}
 \end{enumerate}
 
Property 1 is obtained by rewriting the definition of $\tS_E$. Property 2 is concluded by the connectedness of $\tS$. 
 
 The isolation of stationary points and Property 1 imply that $\tS_E$  only contains finite number of different equivalent classes. Otherwise, there is a sequence of non-equivalent stationary points whose limit is not isolated. Combined the definition of equivalent class and Property 2, we can conclude that $\tS_E$ only contains a single equivalent class; i.e. $\tS_E = \tE(\tA^*)$, where $\tA^*$ is a stationary point of $\tL(\tA)$. Therefore, all the convergent subsequences of $\tA^{(t)}$ converge to one stationary point $\tA^*$ up to equivalence. 
 
 In other words, any iterate $\tA^{(t)}$ generated by Algorithm 1 converges to a stationary point of $\tL(\tA)$ up to equivalence.
 





\end{document}